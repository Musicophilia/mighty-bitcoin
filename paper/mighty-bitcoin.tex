\documentclass[12pt]{article}
\usepackage[margin=1in]{geometry}
% \usepackage{amsmath}
% \usepackage{enumitem}

\usepackage{graphicx,fancyhdr,amsmath,amssymb,amsthm,subfig,url,hyperref, wrapfig, enumitem}
\usepackage[margin=1in]{geometry}

%\newcommand*{\Attack}{A}
\newcommand*{\Time}{T}
\newcommand*{\ABtcOrig}{{B_0}}
%\newcommand*{\ABtcEarned}{{B_{\mathrm{E}}}}
\newcommand*{\ABtcStolen}{{B_{\mathrm{S}}}}
\newcommand*{\ABtcTotal}{{B_{\mathrm{T}}}}
\newcommand*{\NumBtc}{{N_{\mathrm{BTC}}}}
\newcommand*{\BlkReward}{{R}}
\newcommand*{\ExchgRate}{{V_{\mathrm{BTC}}}}
\newcommand*{\AsicValue}{{V_{\mathrm{ASIC}}}}
\newcommand*{\TimeCashOut}{{T_{\mathrm{SellBTC}}}}
\newcommand*{\TimeSellAsics}{{T_{\mathrm{SellASIC}}}}
\newcommand*{\Utility}{U}
\newcommand*{\AttackUtility}{{U_{\mathrm{Attack}}}}
\newcommand*{\PayoffBtc}{{P_{\mathrm{BTC}}}}
\newcommand*{\PayoffAsic}{{P_{\mathrm{ASIC}}}}
\newcommand*{\BlocksPerPeriod}{{N_{\mathrm{B}}}}

\newcommand*{\THs}{\ \mathrm{TH/s}}

\newcommand*{\E}[1]{\mathrm{E}\left[#1\right]}

\newenvironment{defs}
  { \begin{enumerate}[labelindent=0pt,labelwidth=2.5in,itemindent=0em,align=parleft,leftmargin=!] }
  { \end{enumerate} }

\title{Mighty Bitcoin: To Attack Or Not To Attack?}
\author{Karen Wang, Brandon Azad, Lisa Wang, Nishith Khandwala}
\date{\today}

\begin{document}

\maketitle

\section{Introduction}

In class we saw several different attacks against Bitcoin where miners could increase their expected earnings by violating the default protocol. For instance, miners have an incentive to temporarily withhold blocks that they find so that the rest of the network mines on a block that will not be part of the longest chain, reducing the effective hash power of the rest of the Bitcoin network. This attack, known as selfish mining, is profitable to the miner if she controls a sufficiently large proportion of the total hash power.\cite{eyal2014majority}

When Bitcoin was first introduced in 2008, only one mining attack was known: the 51\% attack, in which a miner with 51\% of the hash power could fork the blockchain back to an arbitrary depth.
Despite the potential severity of such an attack,
Satoshi Nakamoto, the pseudonymous inventor of Bitcoin, was not concerned.
He argued that any potential attacker would be better off not attacking, since they can earn more real-world value by playing by the rules:
\begin{quote}
The incentive may help encourage nodes to stay honest. If a greedy attacker is able to assemble more CPU proof-of-worker than all the honest nodes, he would have to choose between using it to defraud people by stealing back his payments, or using it to generate new coins. He ought to find it more profitable to play by the rules, such rules that favour him with more new coins than everyone else combined, than to undermine the system and the validity of his own wealth.\cite{nakamoto2008bitcoin}
\end{quote}

Even after new types mining attacks were discovered, it was still generally believed that Nakamoto's argument held weight: no serious mining attacks against Bitcoin have ever been observed.\footnote{
  The absence of observed mining attacks is not for a lack of profitable opportunity. In June of 2014, the GHash.IO mining pool briefly achieved 51\% of the total mining power. However, rather than risking the possibility of an attack, miners left the pool.\cite{ghashio51}
  More recently, in April of 2016, a transaction with a 291 BTC fee, worth almost $\$137\,000$, was posted to the blockchain.\cite{291btcfee} Despite the large fee, there was no observed attempt to snipe the fee away from the original miner.
} However, the question deserves further study: Is it true that the incentives of the Bitcoin system encourage miners to not attack because attacking would undermine the source of their wealth?
Put another way, are mining attacks profitable if the attacker cares about their holdings in a standard currency like US dollars rather than bitcoin?

To study the profitability of mining attacks in this setting, we have designed and implemented models to simulate the impact of an attack on the value of Bitcoin and evaluate the utility for the attacker.
We have made several simplifying assumptions in order to allow the model to be efficiently described and computed.
Under these assumptions, we evaluated an attacker's expected increase in profit in US dollars by attacking instead of mining cooperatively.

Our experiments showed that for ...  (summarize conclusion / main takeaway)

TODO(???)


\section{Literature Review}

TODO(Karen)

\section{Model Description}

In this section we will describe our assumptions and our model of the attacking miner's utility.
As previously mentioned, we are focusing on mining attacks, not exploits/software bugs that allow the attacker to steal BTC.

\subsection{Assumptions}\label{section:assumptions}

In order to make calculating the utility of a mining attack feasible, we make several strong assumptions.
Most of these assumptions do not hold up in practice.
However, we expect that they are plausible enough to give a somewhat realistic model for how Bitcoin might be affected by an attack.

\begin{enumerate}
  \item
    Future utility is discounted according to a discount rate $\gamma$ per time period.
    Thus, if each time period is a day, one dollar tomorrow is worth $\gamma$ dollars
    today.

    The discount rate actually serves 2 different purposes in our model.
    First, because Bitcoin is highly volatile, the value of one Bitcoin in the future is very uncertain. Thus, the discount factor accomodates one's belief in Bitcoin as a store of value.
    Second, because new and more powerful mining ASICs are introduced regularly, an ASIC is inherently more valuable now than in the future.
  \item
    The mining power of the Bitcoin network is divided among a fixed set of mining
    ASICs. These ASICs are neither created nor destroyed, just sold between miners.
    Thus, the total mining power of the Bitcoin network remains constant.

    This assumption is of course false, but reasonably accurate over the short spans of time (about a month) we are considering.
%   \item
%     Every miner assigns the same price (in USD) to a given ASIC. This price is
%     determined solely by the ASIC's power, the BTC-USD exchange rate, and the
%     discount factor.
  \item
    Every miner values an ASIC with mining power $\beta$ according to the
    expected number of US dollars the ASIC will earn the miner after $L$ time periods,
    assuming:
    \begin{enumerate}
      \item
        the reward for each block is the same known value;
      \item
        blocks are discovered regularly;
      \item
        the ASIC earns the miner a share $\beta$ of the block reward;
      \item
        future utility is discounted by a factor of $\gamma$; and
      \item
        the BTC-USD exchange rate does not change after the time of sale.
    \end{enumerate}

    We will use data on the prices of real ASICs to estimate the value $L$.
  \item
    Only one miner will attack Bitcoin.
  \item
    The miner gets value only in US dollars, not bitcoin. Thus, after the attack, the
    miner will sell the ASICs and bitcoin to gain value in US dollars.
  \item
    If the miner attacks, then the rest of the Bitcoin network will
    immediately detect the attack and know the number of bitcoin ``stolen''
    (that is, the number of bitcoin that were awarded to the attacker due to the attack).
  \item
    The miner already has a fraction $\ABtcOrig$ of the total bitcoin
    before the attack.

    In practice the miner could just sell any bitcoin they own before attacking, causing $\ABtcOrig$ to be $0$. However, if the miner performs many transactions on the bitcoin network, some of those they interact with might prefer operating in bitcoin, which might prevent the miner from cashing out entirely.
  \item
    The miner already has a fraction $\alpha < \frac{1}{2}$ of the total mining power
    before the attack.
  \item
    The miner can spend any earned rewards immediately. That is, we ignore
    the fact that coinbase transactions are locked until they have 100
    confirmations.
  \item
    There are a total of $\NumBtc$ bitcoin in existence, and the current rate
    of mining does not change this value significantly. That is, we assume that
    the total number of bitcoin in existence is constant.

    Again, this assumption is false, but reasonable over the short span of time we are considering.
  \item
    The attack starts at time $\Time = 0$, the miner sells their BTC at time $\TimeCashOut$, and the attacker sells their ASICs at time $\TimeSellAsics$.
\end{enumerate}

\subsection{Model Definition}

We define the following independently tunable variables:

\begin{defs}
  \item[{$\alpha \in (0, \frac{1}{2})$}]
    The mining power of the attacker.
  \item[{$\gamma \in (0, 1]$}]
    The discount rate per time period.
  \item[{$\ABtcOrig \in [0, 1]$}]
    The fraction of all bitcoins owned by the attacker before $T = 0$.
  \item[{$\ABtcStolen
      \in [0, 1 - B_0]$}]
    The fraction of all bitcoins unfairly earned (i.e. ``stolen'') by the miner by attacking. If the miner does not attack, then $\ABtcStolen = 0$.
\end{defs}

We also define the following parameters, which are used in our model but which are not independent input variables:

\begin{defs}
  \item[{$\Time \in [0, \infty)$}]
    The current time period.
  \item[{$\BlocksPerPeriod$}]
    The average number of blocks per time period. If one time period is a day, then $\BlocksPerPeriod \approx 144$.\cite{blksperday}\footnote{
      Bitcoin was designed so that on expectation a block would be mined once every 10 minutes. However, due to natural random variation and bias towards smaller block times due to the slow adjustment of mining difficulty, the average number of blocks per day is often higher than the expected 144. Nonetheless, this effect is small enough not to be relevant for our investigation.
    }
  \item[{$\NumBtc$}]
    The total number of bitcoins in circulation. Currently this is about $16\,000\,000$.\cite{btcincirc}
  \item[{$L$}]
    The mining recuperation period: how long an ASIC needs to run before it earns back its purchase price. We will estimate this value using real-world price data.
  \item[{$\BlkReward$}]
    The block reward in bitcoin. Currently this is 12.5 BTC.\cite{blkreward}
\end{defs}

Finally, we define the following dependent functions and values:

\begin{defs}
%  \item[{$\ABtcEarned(\Attack, \alpha) \in [0, 1 - B_0]$}]
%    The fraction of all bitcoins stolen by the attacker (if Attack = 1) or
%    earned by the attacker (if Attack = 0) at $\Time = 0$.
  \item[{$\ABtcTotal
      = \ABtcOrig + \ABtcStolen$}]
    The total fraction of all bitcoins controlled by the attacker at $\Time = 0$.
  \item[{$\ExchgRate(\ABtcStolen, \Time) \in [0, \infty)$}]
    The BTC-USD exchange rate. More precisely, the value of one bitcoin in US
    dollars at time $\Time$ given that the fraction $\ABtcStolen$ of all
    bitcoins was stolen at time $\Time = 0$.
  \item[{$\AsicValue(\beta, \gamma, \ExchgRate) \in [0, \infty)$}]
    The value in USD of a collection of Bitcoin mining ASICs with mining power $\beta$.
    An explicit formula based on the assumptions in Section \ref{section:assumptions} is given below.
  \item[{$\PayoffBtc(\ABtcTotal, \ExchgRate)
      = \ABtcTotal \cdot \NumBtc \cdot \ExchgRate$}]
    The payout for the miner of cashing in their bitcoins at time $\TimeCashOut$.
  \item[{$\PayoffAsic
      = \AsicValue$}]
    The payout for the miner of selling their ASICs at time $\TimeSellAsics$.
\end{defs}

Under the assumptions stated previously, we can calculate an explicit value for $\AsicValue$. Recall that we assume a miner values an ASIC with mining power $\beta$ according to the expected earnings in dollars after $L$ time periods of mining. % , given that:
% \begin{enumerate}
%     \item each block is worth $R$ bitcoin;
% 	\item blocks are discovered regularly, at a rate of $\BlocksPerPeriod$ per period;
%     \item the ASIC earns the miner a fraction $\beta$ of each block's reward;
%     \item future utility is discounted $\gamma$ per time period; and
%     \item the BTC-USD exchange rate remains fixed.
% \end{enumerate}
If there are $\BlocksPerPeriod$ blocks each time period, each block is worth $\BlkReward$ bitcoin, the ASIC earns the miner a fraction $\beta$ of each block reward, the value of one bitcoin is $\ExchgRate$, and future utility is discounted $\gamma$ per time period, then the expected earnings in US dollars of mining for $L$ time periods is
\begin{align*}
  \AsicValue(\beta, \gamma, \ExchgRate)
  & =
  \sum_{\Time=0}^{L-1} \BlocksPerPeriod \cdot \BlkReward \cdot \beta \cdot \ExchgRate \cdot \gamma^T.
\end{align*}
This simplifies to
\begin{align*}
  \AsicValue(\beta, \gamma, \ExchgRate)
  = \begin{cases}
  \displaystyle
    \BlocksPerPeriod \cdot \BlkReward \cdot \beta \cdot \ExchgRate \cdot \frac{1 - \gamma^L}{1 - \gamma} & \text{if } \gamma < 1 \\
  \displaystyle
    \BlocksPerPeriod \cdot \BlkReward \cdot \beta \cdot \ExchgRate \cdot L & \text{if } \gamma = 1.
  \end{cases}
\end{align*}

Finally, we define the utility function for the miner.
Suppose the miner attacks at $\Time = 0$, cashes out from bitcoins to US dollars at
$\Time = \TimeCashOut$, and sells their ASICs at time $\Time = \TimeSellAsics$.
We define the utility for the miner $\Utility(\alpha, \gamma, \ABtcOrig, \ABtcStolen)$ as the maximum possible discounted earnings in US dollars over all possible values of $\TimeCashOut$ and $\TimeSellAsics$:
\[
  \Utility(\alpha, \gamma, \ABtcOrig, \ABtcStolen) =
  \max_{\substack{\TimeCashOut,\\\TimeSellAsics}}
    \left[
      \begin{aligned}
        & \hphantom{{}+{}}
        \PayoffBtc(\ABtcOrig + \ABtcStolen, \ExchgRate(\ABtcStolen, \TimeCashOut))
          \cdot \gamma^\TimeCashOut \\
        & {}+ \PayoffAsic(\alpha, \gamma, \ExchgRate(\ABtcStolen, \TimeSellAsics))
          \cdot \gamma^\TimeSellAsics
       \end{aligned}
    \right]
\]
Also, we define the utility of attacking over mining as:
\[
  \AttackUtility(\alpha, \gamma, \ABtcOrig, \ABtcStolen) = \Utility(\alpha, \gamma, \ABtcOrig, \ABtcStolen) - \Utility(\alpha, \gamma, \ABtcOrig, 0)
\]
The latter quantity is a measure of the expected profit if the miner attacks and is able to steal $\ABtcStolen$ instead of mining cooperatively.
Thus, the miner has an incentive to attack if and only if $\AttackUtility$ is positive.

\subsection{Fitting Real-World Data}

In order to turn our abstract model into a concrete implementation, we used real-world data to derive values for the above parameters and variables.

We chose each time period to be a day. The values $\BlocksPerPeriod = 144$, $\NumBtc \approx 16\,000\,000$, and $R = 12.5$ are public and depend only on Bitcoin.
This leaves just $L$ and $\ExchgRate$ that have yet to be fully determined.

\subsubsection{Computing ASIC Price Valuation}

In order to estimate $L$, we looked at prices for mining hardware in Bitcoin articles and on eBay, then chose a value of $L$ that made our model for $\AsicValue$ yield the expected prices.
Online, the price for an AntMiner S9 on eBay varies between \$2000 and \$2400, depending on the hashpower.
A typical price for an S9 at $13 \THs$ (terahashes per second, a measure of hashing power in the Bitcoin community) is about \$2349.\cite{ebays9}

As of December 1st, 2016, the Bitcoin network's total hash rate was $2\,052\,749 \THs$, and the price of one bitcoin was \$772.\cite{hashrate, btcprice}
This means that the AntMiner S9 has a mining power of $\beta \approx \frac{13}{2.05\cdot 10^6} \approx 6.33 \cdot 10^{-6}$.
At a discount rate of $\gamma = 0.9999$, this corresponds to $L \approx 271$:
\begin{align*}
  L & = \log_{\gamma} \left(
    1 - \AsicValue \cdot \frac{1 - \gamma}{\BlocksPerPeriod \cdot \BlkReward \cdot \beta \cdot \ExchgRate}
  \right) \\
  & \approx \log_{0.9999} \left(
    1 - 2349 \cdot \frac{1 - 0.9999}{144 \cdot 12.5 \cdot (6.33 \cdot 10^{-6}) \cdot 772}
  \right) \\
  & \approx 270.54.
\end{align*}
We checked whether this value was reasonable under the assumptions of our model by using an online Bitcoin mining profitability calculator called CoinWarz.
The calculator predicted the value $L \approx 262.30$ days to break even.\cite{coinwarz, difficultychart}
The proximity of these two values lends confidence to our choice of ASIC valuation function.
Thus, for the computational model, we chose $L = 270$.

\subsubsection{Computing the BTC to USD Exchange Rate}

From the model description, it is evident that the exchange rate is one of the deciding factors
TODO(Nish)

\begin{enumerate}
  \item
    % http://www.coindesk.com/dao-attacked-code-issue-leads-60-million-ether-theft/
    % https://bitcoinmagazine.com/articles/ethereum-s-dao-forking-crisis-the-bitcoin-perspective-1467404395
    % https://en.wikipedia.org/wiki/The_DAO_(organization)
    % https://www.cryptocoinsnews.com/ex-ethereum-developer-dao-hack-happened-comes-next/
    % https://www.worldcoinindex.com/coin/ethereum
    The Ethereum DAO hack stole about $4\%$ of all ether, and the exchange rate
    almost immediately dropped by almost $50\%$.
    This attack is not directly comparable since it is a hack rather than a
    mining attack.
%  \item
%    On February 24, 2014, an internal document was allegedly leaked showing that Mt.\ Gox had lost $744\,408$ BTC due to theft. Mt.\ Gox closed the next day. The BTC exchange rate dropped from about \$600 to about \$500 over the 24th and 25th, then recovered to \$570 by the 26th. On March 3rd, the price spiked to \$650.
    % http://www.coindesk.com/price/#2014-02-19,2014-02-27,close,bpi,USD
    % http://www.ibtimes.co.uk/mtgox-lost-barely-386-bitcoins-due-cyber-attacks-not-850000-1442242
  \item
    % https://www.theguardian.com/technology/2016/aug/03/bitcoin-stolen-bitfinex-exchange-hong-kong
    On August 3, 2016, the Bitcoin exchange Bitfinex had $120\,000$ BTC stolen, about $0.75\%$ of all Bitcoin in circulation. The value of BTC immediately plunged from \$600 to \$520, then recovered to \$590 by August 7.
%   \item
%     % http://www.coinwarz.com/calculators/bitcoin-mining-calculator/?h=14000.00&p=1350.00&pc=0.25&pf=0.20&d=199312067531.24300000&r=12.50000000&er=577.18000000&hc=2100.00
%     % ROI: 187 days (coinwarz)
%     AntMiner S5: 1.155 Th/s: \$150
%   \item
%     % ROI: 122 days (coinwarz)
%     AntMiner S7: 4.73 Th/s: \$400
%   \item
%     % ROI: just above 1 year (https://www.hobbymining.com/bitmain-antminer-s9/)
%     % ROI: 233 days (coinwarz)
%     AntMiner S9: 13 Th/s: \$2100
  \item
  TODO(Nish): Look at hacks posted in the slack that were used to get the data points used to fit the function.
\end{enumerate}

\section{Experiments}

The primary goal of our project was to determine if and when carrying out a bitcoin mining attack would be profitable and so, using the model proposed in an earlier section, we investigated when the attacker's utility is guaranteed to be positive (or negative). We proposed the following experiment:

Perform a search over the space of model parameters and independent variables and look for maximum positive attacker utility over a set time period. Among the model parameters, we try different values of $B_T$ and among the independent variables, we search over different configurations of $B_0$, $\alpha$ and $\gamma$. In order to simulate possible real world attacks, we limit over search space to reasonably realistic values: $B_0$


\section{Results}

TODO(???): include screenshots here



\section{Limitations of our model}

TODO(???)

\section{Discussion}
What are some other possible reasons that miners have not performed mining attacks on Bitcoin so far?

TODO(Lisa)

\section{Conclusion}

TODO(???)

\bibliographystyle{unsrt}
\bibliography{references}

\end{document}
